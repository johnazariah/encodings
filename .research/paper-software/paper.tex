% FockMap — JOSS Paper
% Compiled with: latexmk -pdf paper.tex
\documentclass[10pt,a4paper]{article}

\usepackage[utf8]{inputenc}
\usepackage[T1]{fontenc}
\usepackage{lmodern}
\usepackage{amsmath,amssymb,amsfonts}
\usepackage{booktabs}
\usepackage{hyperref}
\usepackage{graphicx}
\usepackage{xcolor}
\usepackage{natbib}
\usepackage{authblk}
\usepackage{geometry}
\usepackage{fancyvrb}
\usepackage{listings}

\geometry{margin=1in}
\bibliographystyle{unsrtnat}

% Code listing style for F#
\lstdefinelanguage{FSharp}{
  morekeywords={let,in,if,then,else,match,with,type,of,open,module,fun,rec,mutable,new,true,false,set},
  sensitive=true,
  morecomment=[l]{//},
  morecomment=[s]{(*}{*)},
  morestring=[b]",
}
\lstset{
  language=FSharp,
  basicstyle=\ttfamily\small,
  keywordstyle=\bfseries\color{blue!70!black},
  commentstyle=\itshape\color{green!50!black},
  stringstyle=\color{red!60!black},
  frame=single,
  breaklines=true,
  columns=fullflexible,
}

\hypersetup{
  colorlinks=true,
  linkcolor=blue!70!black,
  citecolor=green!50!black,
  urlcolor=blue!60!black,
}

%%--------------------------------------------------------------------

\title{FockMap: A Composable Functional Framework for\\
  Fermion-to-Qubit Encodings in F\#}

\author[1]{John Azariah\thanks{ORCID: \href{https://orcid.org/0009-0007-9870-1970}{0009-0007-9870-1970}}}
\affil[1]{University of Technology Sydney}

\date{February 2026}

%%--------------------------------------------------------------------
\begin{document}
\maketitle

%%--------------------------------------------------------------------
\section*{Summary}

Simulating fermionic systems on quantum hardware requires a mapping
from fermionic ladder operators to qubit Pauli operators.  This
fermion-to-qubit encoding step strongly influences Pauli weight,
measurement cost, and circuit depth.  Although widely used encodings
(Jordan--Wigner, Bravyi--Kitaev, Parity, and tree-based variants) share
substantial algebraic structure, they are often implemented in existing
software as isolated transformations rather than as instances of a
common formal interface.

\texttt{FockMap} is an open-source F\# library that formalizes this
shared structure through two composable abstractions: \emph{index-set
schemes}, defined by three set-valued functions (\texttt{Update},
\texttt{Parity}, \texttt{Occupation}), and \emph{path-based tree
encodings}, in which any rooted labelled tree induces a valid
encoding.  The index-set abstraction expresses Jordan--Wigner,
Bravyi--Kitaev, and Parity in 3--5 lines each, while the path-based
abstraction supports arbitrary tree topologies, including balanced
binary and balanced ternary trees with optimal $O(\log n)$ asymptotic
Pauli weight.

The operator-processing pipeline is implemented symbolically: Pauli
strings are multiplied exactly with algebraic phase tracking, without
constructing operator matrices and without introducing floating-point
error in intermediate steps.  Numerical coefficients are introduced only
at Hamiltonian assembly.  This design supports transparent correctness
checks and avoids numerical artefacts common in matrix-first workflows.

The library is implemented with algebraic data types and pure functions,
includes a persistent Fenwick tree ADT, and is validated by an extensive
xUnit + FsCheck test suite (416 passing tests), including property-based
algebraic invariants and edge-case regression checks.  Current coverage is
96.55\% line and 87.96\% branch on the core library.  A complete
H$_2$/STO-3G example is provided and reproduced across all five built-in
encodings.

%%--------------------------------------------------------------------
\section*{Statement of Need}

Quantum simulation of molecular electronic structure is widely regarded
as one of the most promising near-term applications of quantum computing
\citep{feynman1982,aspuruguzik2005}.  Between the molecular Hamiltonian
in second quantization and the measurements performed on quantum
hardware lies a critical middleware step: the fermion-to-qubit encoding.
The choice of encoding determines the Pauli weight of each operator (and
hence circuit depth), the number of measurement terms, and ultimately
whether a simulation is feasible on a given device.

Current tools for this step, OpenFermion \citep{mcclean2020}, Qiskit
Nature \citep{qiskit2023}, and PennyLane \citep{bergholm2022},
implement each encoding as a monolithic function mapping
\texttt{FermionOperator} $\to$ \texttt{QubitOperator}.  Adding a new
encoding requires writing hundreds of lines of bespoke code.  The
mathematical structure shared across encodings (Majorana decomposition,
parity tracking, update sets) is duplicated rather than abstracted.
There is no mechanism for users to define, compose, or compare custom
encodings programmatically.

\texttt{FockMap} addresses this gap by representing an encoding as a
\emph{value} (a record of three functions) rather than as an opaque
class hierarchy.  This design enables:

\begin{itemize}
  \item \textbf{Exploration:} researchers can define and test novel
    encodings in 3--5 lines of code.
  \item \textbf{Comparison:} all encodings share the same verification
    pipeline (anti-commutation tests, eigenspectrum comparison).
  \item \textbf{Pedagogy:} the implementation remains close to the
    formal definitions in the literature; an \texttt{EncodingScheme}
    directly encodes the mathematical specification.
\end{itemize}

The library serves quantum computing researchers exploring
encoding-aware circuit synthesis, students learning the algebraic
structure of encodings, and developers building quantum chemistry
simulation pipelines who need a correct and composable encoding layer.

%%--------------------------------------------------------------------
\section*{Functionality}

\subsection*{Encoding Schemes (Index-Set Framework)}

The \texttt{EncodingScheme} record type captures the three index-set
functions that define a fermion-to-qubit encoding.  Three concrete
schemes are provided: \texttt{jordanWignerScheme},
\texttt{bravyiKitaevScheme}, and \texttt{parityScheme}.  User-defined
schemes are first-class values of the same type:

\begin{lstlisting}
let myScheme : EncodingScheme =
    { Update     = fun j n -> set [ j + 1 .. n - 1 ]
      Parity     = fun j   -> if j > 0 then Set.singleton (j - 1)
                               else Set.empty
      Occupation  = fun j   -> if j > 0 then set [j-1; j]
                               else Set.singleton j }
\end{lstlisting}

The framework automatically constructs Majorana operators $c_j$ and
$d_j$ from these three functions, then derives ladder operators
$a^\dagger_j$ and $a_j$ by linear combination.

\subsection*{Tree Encodings (Path-Based Framework)}

Any rooted labelled tree defines a fermion-to-qubit encoding.  The
library provides \texttt{balancedBinaryTree} and
\texttt{balancedTernaryTree} constructors; users can build arbitrary
trees from \texttt{TreeNode} values.  The path-based encoding function
traverses the tree to construct Majorana operators without requiring the
index-set monotonicity constraint, making it strictly more general than
the index-set framework.

\subsection*{Symbolic Algebra Engine}

A distinguishing feature of \texttt{FockMap} is that operator
multiplication is entirely symbolic.  The \texttt{PauliRegister} type
represents a Pauli string (e.g.\ $XZIY$) together with an exact
\texttt{Phase} drawn from $\{+1, -1, +i, -i\}$.  Multiplying two Pauli
registers applies the single-qubit multiplication table
($X \cdot Y = iZ$, etc.)  position-wise and accumulates the phase
algebraically; no $2^n \times 2^n$ matrices are ever constructed.
The resulting \texttt{PauliRegisterSequence} (a weighted sum of Pauli
strings) is the symbolic representation of an encoded operator.

This design means that encoding a ladder operator on 100 modes produces
a compact list of Pauli strings, not a $2^{100} \times 2^{100}$ sparse
matrix.  Correctness can be verified symbolically: the anti-commutation
tests in the verification suite check $\{a_i, a^\dagger_j\} =
\delta_{ij}$ by Pauli string cancellation, not by matrix eigenvalue
comparison.

The library also includes a persistent \texttt{FenwickTree<'a>}
(parameterised over any monoid) and a \texttt{Hamiltonian} module for
constructing molecular Hamiltonians from one-body and two-body integrals.

\subsection*{Verification Suite}

The verification strategy combines unit tests, property-based tests, and
coverage-guided hardening of edge paths.  At submission time, 416 tests pass
for the software package, with 96.55\% line coverage and 87.96\% branch
coverage on the core \texttt{Encodings} library.

Tests cover four categories:
\begin{itemize}
  \item \textbf{Anti-commutation:}
    $\{a_i, a^\dagger_j\} = \delta_{ij}$ verified symbolically for all
    mode pairs.
  \item \textbf{Number conservation:}
    $a^\dagger_j a_j$ produces diagonal Pauli operators.
  \item \textbf{Cross-encoding agreement:} all five encodings produce
    isospectral Hamiltonians for H$_2$ (eigenvalue agreement to
    $5 \times 10^{-16}$).
  \item \textbf{Parser and ordering robustness:} malformed input handling,
    normal/index-order canonicalization, and swap-tracking edge branches for
    deterministic symbolic normalization.
\end{itemize}

These checks provide evidence that symbolic transformations are not only
mathematically correct in the nominal case, but also robust under practical
usage patterns encountered in research code (custom encodings, parser input,
and non-canonical operator orderings).

%%--------------------------------------------------------------------
\section*{Design Principles}

\paragraph{Encodings as data.}
An \texttt{EncodingScheme} is a value, not a class hierarchy.
Jordan--Wigner, Bravyi--Kitaev, and Parity are different values of the
same type.  This enables algebraic reasoning: one can ask whether two
schemes agree on a given mode without running a full encoding.

\paragraph{Two complementary frameworks.}
The index-set framework (\texttt{MajoranaEncoding.fs}) is fast and
algebraically transparent but requires a monotonicity condition on
ancestor indices.  The path-based framework (\texttt{TreeEncoding.fs})
works for \emph{any} tree topology.  Both produce the same output type
(\texttt{PauliRegisterSequence}), so downstream code is
encoding-agnostic.

\paragraph{Symbolic over numerical.}
Existing libraries represent operators as sparse matrices or coefficient
dictionaries indexed by opaque integer keys.  \texttt{FockMap}
represents them as typed Pauli strings with exact algebraic phases.
This makes the intermediate representation human-readable, composable,
and free of floating-point error accumulation.  Numerical coefficients
enter only at the Hamiltonian assembly stage, where they multiply
symbolic Pauli terms.

\paragraph{Pure functions, no mutation.}
All data structures are immutable: persistent Fenwick trees, recursive
tree ADTs, and Pauli register sequences.  The library has zero mutation
and no side effects in its core modules.

\paragraph{Discovered constraints.}
Implementation and testing revealed that the index-set framework's
monotonicity requirement (ancestor indices must exceed descendant
indices) is satisfied only by star-shaped trees, a structural
constraint not previously documented in the literature.  This discovery
motivated the path-based framework as a universal alternative and is
explored further in a companion paper.

%%--------------------------------------------------------------------
\section*{Comparison with Related Software}

\begin{table}[h]
\centering
\small
\begin{tabular}{@{}lccccc@{}}
\toprule
\textbf{Feature} & \textbf{OpenFermion} & \textbf{Qiskit Nature}
  & \textbf{PennyLane} & \textbf{FockMap} \\
\midrule
JW / BK / Parity
  & \checkmark/\checkmark/\checkmark
  & \checkmark/\checkmark/\checkmark
  & \checkmark/\checkmark/---
  & \checkmark/\checkmark/\checkmark \\
Tree encodings & Steiner ext. & --- & --- & Binary, Ternary \\
User-defined encodings & --- & --- & --- & \checkmark \\
User-defined trees & --- & --- & --- & \checkmark \\
Generic encoding abstraction & --- & --- & --- & \checkmark \\
Symbolic Pauli algebra & --- & --- & --- & \checkmark \\
Typed / functional & --- & --- & --- & \checkmark \\
Persistent Fenwick tree & --- & --- & --- & \checkmark \\
\bottomrule
\end{tabular}
\caption{Feature comparison with existing fermion-to-qubit libraries.}
\label{tab:comparison}
\end{table}

OpenFermion \citep{mcclean2020} is the most comprehensive existing tool,
offering extensive support for operator manipulation and circuit
synthesis.  Qiskit Nature \citep{qiskit2023} integrates tightly with IBM
quantum hardware.  PennyLane \citep{bergholm2022} excels at
differentiable quantum computing.  \texttt{FockMap} does not compete on
scope; it provides the \emph{framework} abstraction that these libraries
lack, enabling systematic exploration and comparison of encodings.

%%--------------------------------------------------------------------
\section*{Acknowledgements}

This work is dedicated to Dr.\ Guang Hao Low, whose early encouragement
to study Bravyi--Kitaev encodings motivated the development of this
symbolic algebra framework.  The author also acknowledges the F\#
Software Foundation and the .NET open-source community for the language
and runtime ecosystem supporting this work.

%%--------------------------------------------------------------------
\bibliography{paper}

\end{document}
