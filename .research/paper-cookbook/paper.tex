% FockMap Library Cookbook — Companion Reference
% Compiled with: latexmk -pdf paper.tex
\documentclass[10pt,a4paper]{article}

\usepackage[utf8]{inputenc}
\usepackage[T1]{fontenc}
\usepackage{lmodern}
\usepackage{amsmath,amssymb,amsfonts}
\usepackage{booktabs}
\usepackage{hyperref}
\usepackage{graphicx}
\usepackage{xcolor}
\usepackage{natbib}
\usepackage{authblk}
\usepackage{geometry}
\usepackage{fancyvrb}
\usepackage{listings}
\usepackage{enumitem}
\usepackage{tocloft}

\geometry{margin=1in}
\bibliographystyle{unsrtnat}

% Code listing style for F#
\lstdefinelanguage{FSharp}{
  morekeywords={let,in,if,then,else,match,with,type,of,open,module,fun,rec,
    mutable,new,true,false,set,for,do,to,abstract,member,interface,end,
    and,printfn,Some,None,Array,Map,Seq},
  sensitive=true,
  morecomment=[l]{//},
  morecomment=[s]{(*}{*)},
  morestring=[b]",
}
\lstset{
  language=FSharp,
  basicstyle=\ttfamily\small,
  keywordstyle=\bfseries\color{blue!70!black},
  commentstyle=\itshape\color{green!50!black},
  stringstyle=\color{red!60!black},
  frame=single,
  breaklines=true,
  columns=fullflexible,
  aboveskip=0.8em,
  belowskip=0.6em,
}

\hypersetup{
  colorlinks=true,
  linkcolor=blue!70!black,
  citecolor=green!50!black,
  urlcolor=blue!60!black,
}

%%--------------------------------------------------------------------

\title{FockMap Library Cookbook:\\
  A Progressive Tutorial for Symbolic Fock-Space\\
  Operator Algebra and Fermion-to-Qubit Encodings in F\#}

\author[1]{John Azariah\thanks{ORCID:
  \href{https://orcid.org/0009-0007-9870-1970}{0009-0007-9870-1970}}}
\affil[1]{University of Technology Sydney}

\date{February 2026}

%%--------------------------------------------------------------------
\begin{document}
\maketitle

%%--------------------------------------------------------------------
\begin{abstract}
This document is the companion walkthrough for the FockMap
library~\cite{fockmap2026}, a composable functional framework for
symbolic operator algebra and fermion-to-qubit encodings.  Organized as
13 progressive chapters, it covers every public type, function, and
workflow in the library: from single-qubit Pauli operators through the
three-level algebraic hierarchy (\texttt{C}/\texttt{P}/\texttt{S}),
indexed operators, creation and annihilation, fermionic and bosonic
normal ordering, five built-in encodings, custom encoding schemes,
tree-based encoding via Fenwick and balanced trees, full Hamiltonian
construction, and mixed bosonic--fermionic systems.  Each chapter
introduces concepts through worked F\# code examples that the reader
can execute directly.  The final chapter ties everything together with
a capstone script that encodes the H$_2$ molecule under three encodings
and compares Pauli weight and term count.

This document is also available as hosted Markdown documentation at
the repository website.
\end{abstract}

\tableofcontents
\newpage

%%====================================================================
\section{Hello, Qubit}\label{ch:hello}

Everything in quantum computing eventually becomes a \textbf{Pauli
operator}.  These are $2\times 2$ matrices that act on a single qubit:
%
\begin{equation}
I = \begin{pmatrix}1&0\\0&1\end{pmatrix},\quad
X = \begin{pmatrix}0&1\\1&0\end{pmatrix},\quad
Y = \begin{pmatrix}0&{-i}\\i&0\end{pmatrix},\quad
Z = \begin{pmatrix}1&0\\0&{-1}\end{pmatrix}
\end{equation}

In FockMap they are a simple discriminated union:

\begin{lstlisting}
#r "nuget: FockMap"
open Encodings
open System.Numerics

let identity = I
let bitFlip  = X
let combined = Y
let phase    = Z
\end{lstlisting}

Multiplying two Paulis always yields another Pauli \emph{times a phase}.
The algebra is exact---no floating point involved:

\begin{lstlisting}
let (result, phase) = X * Y
// result = Z,  phase = Pi    because XY = iZ

let (result2, _) = Y * X
// result2 = Z, phase = Mi    because YX = -iZ  (anti-commutation!)

let (result3, _) = X * X
// result3 = I, phase = P1    every Pauli squares to identity
\end{lstlisting}

Notice that $X \cdot Y \neq Y \cdot X$---they differ by a sign.  This
is the \textbf{anti-commutation} property, fundamental to quantum
mechanics.

\subsection{Phases without floating point}

The four phase values $\{+1, -1, +i, -i\}$ live in their own type:

\begin{lstlisting}
// Phase is a discriminated union: P1 (+1), M1 (-1), Pi (+i), Mi (-i)
Pi * Pi      // M1    because i * i = -1
M1 * Mi      // Pi    because (-1) * (-i) = +i
P1 * M1      // M1    the identity doesn't change anything
\end{lstlisting}

When you need to fold a phase into a complex number:

\begin{lstlisting}
let c = Complex(2.0, 0.0)
Pi.FoldIntoGlobalPhase c     // Complex(0.0, 2.0)
M1.FoldIntoGlobalPhase c     // Complex(-2.0, 0.0)
\end{lstlisting}

\textbf{Key insight:} FockMap tracks phases symbolically using
\texttt{Phase} and only converts to floating-point \texttt{Complex} at
the boundaries.  This eliminates the rounding errors that plague na\"ive
Pauli algebra implementations.


%%====================================================================
\section{Building Expressions: The C\,/\,P\,/\,S Hierarchy}\label{ch:cps}

Real quantum operators are \textbf{sums of products} of operators, each
with a coefficient.  FockMap represents this with three nested types:

\begin{center}
\begin{tabular}{lll}
\toprule
\textbf{Type} & \textbf{Role} & \textbf{Analogy} \\
\midrule
\texttt{C<'T>} & Coefficient $\times$ single operator & A letter with emphasis \\
\texttt{P<'T>} & Product of operators & A word (ordered sequence) \\
\texttt{S<'T>} & Sum of products & A sentence \\
\bottomrule
\end{tabular}
\end{center}

These types are \textbf{generic}---they work with any operator type.

\subsection{C --- a single weighted operator}

\begin{lstlisting}
let one_x = C<Pauli>.Apply X              // 1 * X
let half_y = C<Pauli>.Apply(Complex(0.5, 0.0), Y)
\end{lstlisting}

\subsection{P --- an ordered product (tensor product)}

\begin{lstlisting}
let xy = one_x * half_y
// P<Pauli> with Coeff = 0.5, Units = [X; Y]

let xzy = P<Pauli>.Apply [| X; Z; Y |]
// 1 * (X (x) Z (x) Y)

let scaled = xzy.ScaleCoefficient(Complex(3.0, 0.0))
// 3 * (X (x) Z (x) Y)
\end{lstlisting}

Reduction normalises internal coefficients into the single overall
coefficient:

\begin{lstlisting}
let mixed = P<Pauli>.Apply(Complex(2.0, 0.0), [| half_y; one_x |])
let clean = mixed.Reduce.Value
// Coeff = 1.0, Units = [Y; X]
\end{lstlisting}

\subsection{S --- a sum of products (the Hamiltonian shape)}

\begin{lstlisting}
let s1 = S<Pauli>.Apply(P<Pauli>.Apply [| X; Z |])
let s2 = S<Pauli>.Apply(P<Pauli>.Apply [| Y; I |])

let hamiltonian = s1 + s2
let doubled = s1 + s1    // 2*(X (x) Z)
\end{lstlisting}

\texttt{S<'T>} stores its terms in a \texttt{Map<string, P<'T>>}, keyed
by string representation.  Like terms combine automatically: this is
what makes \texttt{s1 + s1} produce $2 \cdot (X \otimes Z)$ instead of
two separate entries.

\subsection{Coefficient hygiene and zero propagation}

Every level has a \texttt{Reduce} method that replaces \texttt{NaN} and
infinity with zero, preventing numerical corruption from propagating.
A product containing any zero-coefficient unit becomes the zero product
eagerly, so downstream code never wastes time on trivial terms.


%%====================================================================
\section{Operators on Specific Qubits}\label{ch:ixop}

The \texttt{IxOp} type tags each operator with a mode index:

\begin{lstlisting}
let x0 = IxOp<uint32, Pauli>.Apply(0u, X)   // "X on qubit 0"
let z3 = IxOp<uint32, Pauli>.Apply(3u, Z)   // "Z on qubit 3"
\end{lstlisting}

\subsection{Parsing from strings}

\begin{lstlisting}
let parsed = Pauli.FromString "(X, 2)"

let term = PIxOp<uint32, Pauli>.TryCreateFromString
               Pauli.Apply "[(X, 0)|(Z, 3)]"

let expr = SIxOp<uint32, Pauli>.TryCreateFromString
               Pauli.Apply "{[(X, 0)|(Z, 1)]; [(Y, 0)|(I, 1)]}"
\end{lstlisting}

The format uses \texttt{[...|...]} for products and \texttt{\{...;~...\}}
for sums.  You pass a parser function for the underlying operator type.


%%====================================================================
\section{Creation and Annihilation}\label{ch:ladder}

Quantum chemistry works with \textbf{ladder operators}: creation
($a^\dagger$) and annihilation ($a$):

\begin{lstlisting}
let create  = Raise     // a+
let destroy = Lower     // a
let nothing = Identity  // I
\end{lstlisting}

\subsection{Indexed ladder operators}

\begin{lstlisting}
let adag2 = LadderOperatorUnit.FromUnit(true, 2u)   // a+_2
let a1    = LadderOperatorUnit.FromUnit(false, 1u)   // a_1
\end{lstlisting}

\subsection{Product terms}

A typical quantum chemistry term: $a^\dagger_0 a^\dagger_1 a_1 a_0$:

\begin{lstlisting}
let twoBody = LadderOperatorProductTerm.FromUnits [|
    (true,  0u); (true,  1u); (false, 1u); (false, 0u)
|]

twoBody.IsInNormalOrder    // true
twoBody.IsInIndexOrder     // true
\end{lstlisting}


%%====================================================================
\section{Normal Ordering: Making Physics Legal}\label{ch:normal}

The central problem: we must put operator expressions in \textbf{normal
order}---all creation operators before all annihilation operators.

Fermions obey the canonical anti-commutation relations (CAR):
%
\begin{equation}
\{a_i, a^\dagger_j\} = a_i a^\dagger_j + a^\dagger_j a_i = \delta_{ij}
\end{equation}

\begin{lstlisting}
let disordered =
    P<IxOp<uint32, LadderOperatorUnit>>.Apply [|
        LadderOperatorUnit.FromUnit(false, 0u)   // a_0
        LadderOperatorUnit.FromUnit(true, 1u)     // a+_1
    |]
    |> S<IxOp<uint32, LadderOperatorUnit>>.Apply

let ordered =
    LadderOperatorSumExpr<FermionicAlgebra>.ConstructNormalOrdered
        disordered
// Result: -1 * a+_1 a_0  (sign flipped!)
\end{lstlisting}

Same-index operators generate an identity term:
%
\begin{equation}
a_0 a^\dagger_0 = \delta_{00} - a^\dagger_0 a_0 = 1 - a^\dagger_0 a_0
\end{equation}

\subsection{The algebra is pluggable}

Commutation relations live behind an interface:

\begin{lstlisting}
type ICombiningAlgebra<'op> =
    abstract Combine :
        P<IxOp<uint32,'op>>
            -> C<IxOp<uint32,'op>>
            -> P<IxOp<uint32,'op>>[]
\end{lstlisting}

\begin{center}
\begin{tabular}{llll}
\toprule
\textbf{Algebra} & \textbf{Class} & \textbf{Physics} & \textbf{Key behaviour} \\
\midrule
Fermionic & \texttt{FermionicAlgebra} & Electrons & Swap $\Rightarrow$ sign flip \\
Bosonic   & \texttt{BosonicAlgebra}   & Photons, phonons & Swap $\Rightarrow$ no sign \\
\bottomrule
\end{tabular}
\end{center}

\subsection{Bosonic normal ordering}

Compare $b_0 b^\dagger_0$ under bosonic algebra:

\begin{lstlisting}
let bosonicResult = constructBosonicNormalOrdered bosonicExpr
// Result: 1 + b+_0 b_0   (PLUS sign --- bosons commute)
\end{lstlisting}

Fermionic: $a_0 a^\dagger_0 = 1 - a^\dagger_0 a_0$.
Bosonic: $b_0 b^\dagger_0 = 1 + b^\dagger_0 b_0$.
That single sign difference distinguishes matter from light.


%%====================================================================
\section{Your First Encoding}\label{ch:first}

The problem: quantum computers have qubits, but chemistry uses fermions.
We need a mapping.

\subsection{Jordan--Wigner}

The simplest encoding~\cite{jordanwigner1928} inserts a chain of $Z$
operators on all preceding qubits:

\begin{lstlisting}
let result = jordanWignerTerms Raise 2u 4u

for term in result.DistributeCoefficient.SummandTerms do
    printfn "%s %s" term.PhasePrefix term.Signature
//  0.5 ZZXI
// -0.5i ZZYI
\end{lstlisting}

The result is $a^\dagger_2 = \tfrac{1}{2}(ZZXI) -
\tfrac{i}{2}(ZZYI)$.

\subsection{The Z-chain problem}

The Pauli weight grows \textbf{linearly}---$O(n)$.  For 100 qubits the
last operator touches all 100 qubits.


%%====================================================================
\section{Five Encodings, One Interface}\label{ch:five}

Every encoding function has the same type signature:

\begin{lstlisting}
type EncoderFn =
    LadderOperatorUnit -> uint32 -> uint32
        -> PauliRegisterSequence
\end{lstlisting}

This makes them drop-in replacements:

\begin{lstlisting}
let jw  = jordanWignerTerms       Raise mode n   // O(n)
let bk  = bravyiKitaevTerms       Raise mode n   // O(log n)
let par = parityTerms             Raise mode n   // O(n)
let bt  = balancedBinaryTreeTerms Raise mode n   // O(log n)
let tt  = ternaryTreeTerms        Raise mode n   // O(log_3 n)
\end{lstlisting}

\subsection{PauliRegister and PauliRegisterSequence}

Every encoding returns a \texttt{PauliRegisterSequence}---a sum of
\texttt{PauliRegister} terms:

\begin{lstlisting}
let reg = PauliRegister("ZZXI", Complex.One)
reg.Signature       // "ZZXI"
reg.Coefficient     // Complex(1.0, 0.0)
reg.[0]             // Some Z
\end{lstlisting}


%%====================================================================
\section{How Encodings Work Under the Hood}\label{ch:internals}

\subsection{Majorana decomposition}

A ladder operator is split into two Majorana operators:
%
\begin{align}
a^\dagger_j &= \tfrac{1}{2}(c_j - i\,d_j), &
a_j &= \tfrac{1}{2}(c_j + i\,d_j)
\end{align}

The Majorana operators are built from three index sets:
%
\begin{align}
c_j &= X_{U(j) \cup \{j\}} \cdot Z_{P(j)} \\
d_j &= Y_j \cdot X_{U(j)} \cdot Z_{(P(j) \oplus \text{Occ}(j)) \setminus \{j\}}
\end{align}

\subsection{The EncodingScheme record}

\begin{lstlisting}
type EncodingScheme =
    { Update     : int -> int -> Set<int>    // U(j, n)
      Parity     : int -> Set<int>           // P(j)
      Occupation : int -> Set<int> }         // Occ(j)
\end{lstlisting}

\subsection{Custom encoding in 5 lines}

\begin{lstlisting}
let myJW : EncodingScheme =
    { Update     = fun _ _ -> Set.empty
      Parity     = fun j   -> set [ for k in 0 .. j-1 -> k ]
      Occupation = fun j   -> set [j] }

let myResult = encodeOperator myJW Raise 2u 4u
// Identical to jordanWignerTerms Raise 2u 4u!
\end{lstlisting}

Compare that to the $\sim$200 lines needed in other
frameworks~\cite{mcclean2020}.

\subsection{Inspecting Majorana assignments}

\begin{lstlisting}
let cAssign = cMajorana jordanWignerScheme 2 4
// [(0, Z); (1, Z); (2, X)]

let dAssign = dMajorana jordanWignerScheme 2 4
// [(0, Z); (1, Z); (2, Y)]

let cReg = pauliOfAssignments 4 cAssign Complex.One
// PauliRegister("ZZXI", 1.0)
\end{lstlisting}


%%====================================================================
\section{Trees and Fenwick Trees}\label{ch:trees}

\subsection{Why trees?}

The Z-chain in Jordan--Wigner grows linearly because it uses a linear
data structure.  A \textbf{tree} shares parity information, cutting
depth to $O(\log n)$.

\subsection{Fenwick Trees}

The Bravyi--Kitaev encoding~\cite{bravyikitaev2002,seeley2012} is built
on a Fenwick tree.  FockMap provides a purely functional implementation:

\begin{lstlisting}
let tree = FenwickTree.ofArray (^^^) 0 occupations

FenwickTree.prefixQuery tree 3
FenwickTree.pointQuery tree 5
let tree' = FenwickTree.update tree 2 0
\end{lstlisting}

\subsection{Encoding trees}

\begin{lstlisting}
let linear  = linearTree 8            // Jordan-Wigner
let binary  = balancedBinaryTree 8     // O(log_2 n)
let ternary = balancedTernaryTree 8    // O(log_3 n)
\end{lstlisting}

\subsection{Two frameworks}

\textbf{Framework 1} (index sets, Fenwick-compatible):
\begin{lstlisting}
let scheme = treeEncodingScheme (balancedBinaryTree 8)
encodeOperator scheme Raise 2u 8u
\end{lstlisting}

\textbf{Framework 2} (path-based, any ternary
tree)~\cite{jiang2020}:
\begin{lstlisting}
let result = encodeWithTernaryTree tree Raise 2u 8u
\end{lstlisting}


%%====================================================================
\section{Building a Real Hamiltonian}\label{ch:hamiltonian}

The electronic Hamiltonian in second quantization:
%
\begin{equation}
H = \sum_{pq} h_{pq}\, a^\dagger_p a_q
  + \tfrac{1}{2} \sum_{pqrs} \langle pq|rs\rangle\,
    a^\dagger_p a^\dagger_q a_s a_r
\end{equation}

\subsection{Step 1 --- Define integrals}

For H$_2$ in STO-3G (4 spin-orbitals):

\begin{lstlisting}
let nModes = 4u
let oneBody = Map [
    ("00", Complex(-1.2563, 0.0))
    ("11", Complex(-1.2563, 0.0))
    ("22", Complex(-0.4719, 0.0))
    ("33", Complex(-0.4719, 0.0))
]
\end{lstlisting}

\subsection{Step 2 --- Coefficient lookup}

\begin{lstlisting}
let lookup (key : string) =
    match key.Length with
    | 2 -> oneBody |> Map.tryFind key
    | 4 -> twoBody |> Map.tryFind key
    | _ -> None
\end{lstlisting}

\subsection{Step 3 --- Compute}

\begin{lstlisting}
let hamiltonian = computeHamiltonian lookup nModes
\end{lstlisting}

\subsection{Step 4 --- Swap the encoding}

\begin{lstlisting}
let hBK = computeHamiltonianWith bravyiKitaevTerms lookup nModes
let hTT = computeHamiltonianWith ternaryTreeTerms  lookup nModes
\end{lstlisting}

All three Hamiltonians have the same eigenvalues.


%%====================================================================
\section{Mixed Bosonic--Fermionic Systems}\label{ch:mixed}

Some models combine fermions and bosons (e.g.\ electron--phonon
coupling).

\subsection{Sector tagging}

\begin{lstlisting}
let f0_up   = fermion Raise 0u    // f+_0
let b1_down = boson Lower 1u     // b_1
\end{lstlisting}

\subsection{Canonical block order}

Cross-sector commutators are zero:
%
\begin{equation}
[a_i, b_j] = [a_i, b_j^\dagger]
           = [a_i^\dagger, b_j]
           = [a_i^\dagger, b_j^\dagger] = 0
\end{equation}

The canonical form places all fermionic operators left and bosonic
operators right.

\subsection{Full mixed normal ordering}

\texttt{constructMixedNormalOrdered} performs three steps:
\begin{enumerate}[nosep]
  \item Sector ordering (fermions left, bosons right; no sign change).
  \item Fermionic normal ordering (CAR; sign flips).
  \item Bosonic normal ordering (CCR; no sign flips).
\end{enumerate}

\begin{lstlisting}
let result = constructMixedNormalOrdered messyExpr
\end{lstlisting}

\subsection{Decision guide}

\begin{enumerate}[nosep]
  \item No bosons? Standard fermionic path.
  \item Bosons but no qubit mapping yet? Hybrid pipeline.
  \item Choosing an encoding? Compare on extracted fermion blocks.
  \item Cutoff-sensitive? Convergence checks first.
\end{enumerate}

\subsection{Common failure modes}

\begin{center}
\small
\begin{tabular}{lll}
\toprule
\textbf{Symptom} & \textbf{Cause} & \textbf{Fix} \\
\midrule
Unexpected sign flips & Cross-sector/fermionic confusion & Canonicalise first \\
Non-deterministic shape & No block rule & \texttt{constructMixedNormalOrdered} \\
Bloated encoding & Identity placeholders & Drop identities first \\
Hard-to-debug modes & Implicit sectoring & Explicit constructors \\
Unstable bosonic results & Cutoff too low & Sweep and compare \\
\bottomrule
\end{tabular}
\end{center}


%%====================================================================
\section{The Utility Belt}\label{ch:utils}

\subsection{Complex number extensions}

\begin{lstlisting}
let c = Complex(1.0, 2.0)
c.IsFinite; c.IsNonZero; c.TimesI; c.Reduce

Complex.SwapSignMultiple 3 Complex.One    // -1
Complex.SwapSignMultiple 4 Complex.One    // +1
\end{lstlisting}

\subsection{Map extensions}

\begin{lstlisting}
let m = Map [ ("a", 1); ("b", 2) ]
m.Keys; m.Values
\end{lstlisting}

\subsection{Currying utilities}

\begin{lstlisting}
let addTupled = uncurry add      // (int * int) -> int
let addCurried = curry addTupled // int -> int -> int
\end{lstlisting}


%%====================================================================
\section{Grand Finale: Three Encodings, One Molecule}\label{ch:finale}

This script ties every chapter together, encoding H$_2$ with three
different encodings and comparing the results:

\begin{lstlisting}
open Encodings
open System.Numerics

let nModes = 4u

let integrals = Map [
    ("00", Complex(-1.2563, 0.0))
    ("11", Complex(-1.2563, 0.0))
    ("22", Complex(-0.4719, 0.0))
    ("33", Complex(-0.4719, 0.0))
    (* two-body integrals omitted for brevity *)
]

let lookup key =
    match (key : string).Length with
    | 2 | 4 -> integrals |> Map.tryFind key
    | _ -> None

let encoders = [
    ("Jordan-Wigner",  jordanWignerTerms)
    ("Bravyi-Kitaev",  bravyiKitaevTerms)
    ("Ternary Tree",   ternaryTreeTerms)
]

for (name, encoder) in encoders do
    let ham = computeHamiltonianWith encoder lookup nModes
    let terms = ham.DistributeCoefficient.SummandTerms
    let avgWeight =
        terms
        |> Array.averageBy (fun t ->
            t.Signature
            |> Seq.filter (fun c -> c <> 'I')
            |> Seq.length |> float)
    printfn "%s: %d terms, avg weight %.2f"
        name terms.Length avgWeight
\end{lstlisting}

All three Hamiltonians have the same eigenvalues---they represent
identical physics.  The differences in term count and Pauli weight
affect measurement cost on real quantum hardware.


%%====================================================================
\section*{Quick Reference}\label{sec:quickref}

\subsection*{Encoding functions}

\begin{center}
\begin{tabular}{lcl}
\toprule
\textbf{Function} & \textbf{Scaling} & \textbf{Best for} \\
\midrule
\texttt{jordanWignerTerms} & $O(n)$ & Small systems \\
\texttt{bravyiKitaevTerms} & $O(\log_2 n)$ & General purpose \\
\texttt{parityTerms} & $O(n)$ & Parity-natural basis \\
\texttt{balancedBinaryTreeTerms} & $O(\log_2 n)$ & Binary tree \\
\texttt{ternaryTreeTerms} & $O(\log_3 n)$ & Best asymptotic \\
\texttt{encodeOperator} & Varies & Custom scheme \\
\texttt{encodeWithTernaryTree} & Varies & Custom tree \\
\bottomrule
\end{tabular}
\end{center}

\subsection*{Type cheat sheet}

\begin{center}
\small
\begin{tabular}{ll}
\toprule
\textbf{Type} & \textbf{Represents} \\
\midrule
\texttt{Pauli} & $I$, $X$, $Y$, $Z$ \\
\texttt{Phase} & Exact phase: $+1$, $-1$, $+i$, $-i$ \\
\texttt{C<'T>} & Coefficient $\times$ operator \\
\texttt{P<'T>} & Ordered product \\
\texttt{S<'T>} & Sum of products \\
\texttt{IxOp<'idx,'op>} & Indexed operator \\
\texttt{LadderOperatorUnit} & $a^\dagger$ / $a$ / $I$ \\
\texttt{PauliRegister} & Pauli string + coefficient \\
\texttt{PauliRegisterSequence} & Sum of Pauli strings \\
\texttt{EncodingScheme} & Three index-set functions \\
\texttt{EncodingTree} & Tree shape \\
\texttt{FenwickTree<'a>} & Immutable binary indexed tree \\
\texttt{SectorLadderOperatorUnit} & Sector-tagged ladder op \\
\bottomrule
\end{tabular}
\end{center}


%%--------------------------------------------------------------------
\section*{Acknowledgements}

This work is dedicated to Dr.\ Guang Hao Low, whose encouragement to
study Bravyi--Kitaev encodings motivated the development of this
framework.

%%--------------------------------------------------------------------
\bibliography{paper}
\end{document}
