% drafts/07-discussion.tex — Discussion
%
% Target length: ~2.5 pages (two-column PRA)
% Subsections: 7.1 Holographic map, 7.2 Hamiltonian-adapted trees,
%              7.3 Noise-aware encodings, 7.4 Open questions

%%====================================================================
\subsection{The encoding as a holographic map}
\label{sec:disc-holo}

The tree--encoding correspondence is not merely a formal analogy: it
shares structural features with holographic codes
\cite{pastawski2015happy,swingle2012}.

\begin{center}
\begin{tabular}{@{}ll@{}}
\toprule
\textbf{Holographic code} & \textbf{Fermion-to-qubit encoding} \\
\midrule
Bulk Hilbert space   & Fock space (fermionic) \\
Boundary Hilbert space & Qubit space \\
Tensor network / code  & Labelled rooted tree \\
Bulk local operator  & Ladder operator $\ad{j}$ \\
Boundary representation & Pauli string $E(\ad{j})$ \\
Emergent locality    & Weight bound (\cref{thm:weight-bound}) \\
Gauge constraints    & Stabiliser group \\
\bottomrule
\end{tabular}
\end{center}

\paragraph{What carries over.}
(i)~Locality emergence: a bulk-local operator may be boundary-local or
boundary-non-local depending on the code.
(ii)~Symmetry: a bulk symmetry maps to different boundary
representations.
(iii)~Gauge constraints: bulk constraints become boundary stabilisers.

\paragraph{What does not carry over.}
(i)~No gravitational dynamics — the encoding is a static map.
(ii)~No entanglement wedge reconstruction — all bulk operators are
reconstructible.
(iii)~The Hilbert space is finite-dimensional; there is no UV/IR
hierarchy in the usual sense.

Despite these differences, the encoding provides a \emph{calculable}
toy model for phenomena that are intractable in the full AdS/CFT
setting.

%%====================================================================
\subsection{Hamiltonian-adapted trees}
\label{sec:disc-adapted}

The results of \cref{sec:emergence} suggest that the optimal tree
depends on the Hamiltonian being encoded.  For a nearest-neighbour
fermionic chain, the Jordan--Wigner encoding gives weight-2 hopping
terms — outperforming the balanced ternary tree despite its worse
worst-case scaling.  For non-local Hamiltonians, the balanced ternary
tree's $O(\log_3 n)$ bound is near-optimal.

Given a Hamiltonian $H$ and a hardware connectivity graph~$G$, the
problem of finding the tree $T$ that minimises the resulting CNOT count
is likely NP-hard.  Approximation algorithms and heuristic searches
\cite{loaiza2023,goings2023} offer a practical path forward.

%%====================================================================
\subsection{Noise-aware encodings}
\label{sec:disc-noise}

On current quantum hardware, not all qubit pairs have equal gate
fidelity.  An encoding that routes high-weight operators through
low-noise qubit subsets could outperform a theoretically optimal
encoding.  The tree provides a natural parametrisation for this
optimisation: tree mutations (edge swaps, re-rootings) define a search
landscape over encodings.

%%====================================================================
\subsection{Open questions}
\label{sec:disc-open}

We conclude with several directions for future work.

\begin{enumerate}
  \item \textbf{Entanglement entropy of the encoding.}
    Different encodings produce different entanglement in the qubit
    computational basis for the same fermionic ground state.  Can this
    ``encoding entanglement'' be defined intrinsically?

  \item \textbf{Topological invariants of the encoding space.}
    The space of trees has a graph structure via edge swaps.  Does
    this graph possess topological features that distinguish encoding
    classes?

  \item \textbf{Field-theoretic interpretation of the phase boundary.}
    Is there a continuum limit in which the monotonicity boundary
    becomes a phase transition in the statistical-mechanical sense?

  \item \textbf{Complexity of optimal-tree search.}
    What is the computational complexity of finding the weight-minimising
    tree for a given Hamiltonian?

  \item \textbf{Composition with error correction.}
    An encoding followed by an error-correcting code produces a
    composite map.  What properties of the encoding survive error
    correction, and what new structure appears?
\end{enumerate}
