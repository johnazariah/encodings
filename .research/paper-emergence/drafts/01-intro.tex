% drafts/01-intro.tex — Introduction
%
% Target length: ~2.5 pages (two-column PRA)
% Dependencies: none

Quantum simulation of fermionic systems is one of the most promising
near-term applications of quantum computers.  Yet every such simulation
must cross an algebraic divide: electrons obey anticommutation relations,
while the native operations of a qubit processor are products of Pauli
matrices that commute across distinct sites.  The bridge between the two
is a \emph{fermion-to-qubit encoding} — a mapping that translates
creation and annihilation operators into Pauli strings while faithfully
preserving the canonical anticommutation relations (CAR).

A substantial body of work has studied specific encodings
\cite{jordan1928,bravyi2002,seeley2012,tranter2015}, compared their
circuit-level costs \cite{havlicek2017}, and unified them within
tree-based frameworks \cite{steudtner2018,jiang2020,miller2023bonsai}.
In all of this, the encoding is treated as a \emph{tool} — a means to
an end.  The question that drives the present paper is different:

\begin{quote}
\itshape
What does the encoding \textbf{mean}?
\end{quote}

We argue that a fermion-to-qubit encoding is a minimal, exactly solvable
model of \emph{quantum emergence}.  The fermionic Hamiltonian and its
qubit image share an identical eigenspectrum, yet every structural
property — locality, symmetry representation, gauge structure,
computational complexity — differs between the two descriptions.
These differences are not artefacts to be optimised away; they are
\emph{emergent} properties, created by the act of re-representation.



%--- The tree--encoding connection (brief) ---
\subsection*{The tree is the encoding}

Our central result is that every CAR-preserving encoding of $n$
fermionic modes is uniquely specified by a labelled rooted tree on $n$
nodes — and conversely, every such tree defines a valid encoding.  The
familiar Jordan--Wigner, Bravyi--Kitaev, and Parity encodings
correspond to particular tree shapes (linear chain, Fenwick tree,
reverse chain); balanced binary and ternary trees yield encodings with
$O(\log n)$ Pauli weight.  The number of distinct encodings on $n$
modes is $n^{n-1}$ (Cayley's formula) — an enormous space whose
structural properties we characterise for the first time.

%--- Emergence angle ---
\subsection*{Emergence from representation}

From the tree alone — without solving any physics — one can read off:
\begin{enumerate}
  \item the Pauli weight of every encoded operator (emergent locality),
  \item how the global $\mathbb{Z}_2$ parity symmetry is represented on
        qubits (emergent symmetry fractionalization),
  \item what stabiliser constraints appear when fixing particle number
        (emergent gauge structure),
  \item the renormalization hierarchy implicit in the encoding
        (the tree as a coarse-graining scheme).
\end{enumerate}
None of these properties exist in the fermionic description.  They are
created by the encoding, and they depend entirely on the tree.  This
makes the encoding a concrete, finite-dimensional analogue of emergence
phenomena studied in condensed matter~\cite{anderson1972more,wen2017zoo}
and holography~\cite{swingle2012,pastawski2015happy}.

%--- Phase boundary ---
\subsection*{A structural phase transition}

The space of labelled rooted trees decomposes into two regions.
\emph{Index-monotonic} trees — those in which every ancestor's label
exceeds its descendant's — admit a compact algebraic construction
(the index-set method of Seeley--Richard--Love~\cite{seeley2012}).
\emph{Non-monotonic} trees require the geometric, path-based
construction of Jiang et al.~\cite{jiang2020}.  We prove that the
index-set method silently violates the CAR on non-monotonic trees,
identify the precise failure mechanism, and show that the monotonic
fraction vanishes as $n$ grows.  This boundary is itself a structural
phase transition in the space of representations.

%--- Outline ---
\subsection*{Outline}

\Cref{sec:background} reviews the fermionic and Pauli algebras.
\Cref{sec:tree-encoding} establishes the tree--encoding correspondence
via two constructions and recovers every known encoding as a special
case.
\Cref{sec:emergence} identifies emergent locality, symmetry
fractionalization, gauge structure, and renormalization as consequences
of the tree geometry.
\Cref{sec:phase-boundary} characterises the monotonicity boundary.
\Cref{sec:validation} presents computational verification.
\Cref{sec:discussion} draws connections to holography, Hamiltonian-adapted
trees, and open questions.
