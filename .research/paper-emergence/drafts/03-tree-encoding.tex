% drafts/03-tree-encoding.tex — The Tree–Encoding Correspondence
%
% Target length: ~4 pages (two-column PRA)
% Subsections: 3.1 Labelled rooted trees, 3.2 Construction A (index-set),
%              3.3 Construction B (path-based), 3.4 Recovering known encodings

This section establishes the central mathematical result: every
CAR-preserving encoding corresponds to a labelled rooted tree, and
every tree defines an encoding.

%%====================================================================
\subsection{Labelled rooted trees}
\label{sec:te-trees}

\begin{definition}
A \emph{labelled rooted tree} $T$ on $[n] = \{0, 1, \ldots, n{-}1\}$
is a tree with $n$ vertices, each labelled by a distinct element of
$[n]$, with a distinguished root vertex.
\end{definition}

The number of labelled rooted trees on $n$ vertices is $n^{n-1}$ by
Cayley's formula~\cite{cayley1889}.  This is the cardinality of the
encoding space.

For a vertex $v$ in $T$, we write $\depth{v}$ for the length of the
unique root-to-$v$ path.

%%====================================================================
\subsection{Construction A: the index-set method}
\label{sec:te-indexset}

Following Seeley, Richard, and Love~\cite{seeley2012}, we define for
each node $j$ the following index sets derived from $T$:
%
\begin{align}
  \Uset{j} &= \{\text{ancestors of } j\}, \label{eq:update-set} \\
  \Rset{j} &= \{v : v \text{ is a child of an ancestor of } j,\;
               v < j,\; v \notin \Uset{j}\}, \\
  \Occ{j}  &= \{j\} \cup \{\text{descendants of } j\}.
\end{align}
%
Let $F(j)$ denote the children of $j$ and
$\Pset{j} = \Rset{j} \cup F(j)$.  The Majorana operators are then
%
\begin{align}
  c_j &= X_{\Uset{j} \cup \{j\}} \cdot Z_{\Pset{j}},
    \label{eq:cj-indexset} \\
  d_j &= Y_j \cdot X_{\Uset{j}} \cdot
         Z_{(\Pset{j} \oplus \Occ{j}) \setminus \{j\}}.
    \label{eq:dj-indexset}
\end{align}

\textbf{Critical restriction.}  This construction satisfies the CAR
\emph{only} when the tree is index-monotonic (\cref{sec:phase-boundary}).

%%====================================================================
\subsection{Construction B: the path-based method}
\label{sec:te-pathbased}

The path-based construction \cite{jiang2020,miller2023bonsai} is
universal — it works for \emph{any} labelled rooted tree with branching
factor at most~3.

\paragraph{Link labelling.}
Each internal node $v$ has up to three descending connections.  Label
them $X$, $Y$, $Z$.  A connection leading to a child is an
\emph{edge}; one with no child is a \emph{leg} (a free Majorana
endpoint).

\paragraph{Majorana strings.}
The Majorana operator for a leg $\ell$ is the product of Pauli labels
along the root-to-$\ell$ path:
%
\begin{equation}
  S_\ell = \prod_{(v,\,\lambda) \in \mathrm{path}(\text{root},\,\ell)}
           \lambda_v,
  \label{eq:majorana-path}
\end{equation}
%
where $\lambda_v \in \{X_v, Y_v, Z_v\}$ is the Pauli corresponding to
the label of the link taken at vertex~$v$.

\paragraph{Leg pairing.}
For each vertex $v$:
\begin{itemize}
  \item Follow the $X$-link, then $Z$-links until a leg
        $\to$ Majorana $c_v$.
  \item Follow the $Y$-link, then $Z$-links until a leg
        $\to$ Majorana $d_v$.
\end{itemize}

\begin{proposition}
\label{prop:pathbased-car}
For any rooted tree $T$ on $[n]$ with branching factor $\le 3$,
Construction~B produces $2n$ Majorana operators satisfying the CAR.
\end{proposition}

\begin{proof}[Proof sketch]
Two Majorana strings $S_\ell$ and $S_{\ell'}$ for distinct legs
$\ell, \ell'$ share a common prefix from the root to their lowest
common ancestor, then diverge at different Pauli labels on the same
qubit.  At the divergence point, the distinct labels guarantee
anticommutation; the shared prefix contributes only an overall sign.
The full proof proceeds by induction on the tree depth.
\end{proof}

%%====================================================================
\subsection{Recovering known encodings}
\label{sec:te-recovery}

\begin{table}[t]
\centering
\caption{Known encodings as special trees.}
\label{tab:encodings-as-trees}
\begin{tabular}{@{}lcccc@{}}
\toprule
\textbf{Tree shape} & \textbf{Depth} & \textbf{Constr.} &
  \textbf{Encoding} & \textbf{Weight} \\
\midrule
Linear chain        & $n{-}1$              & A or B & JW      & $O(n)$ \\
Fenwick tree        & $\lfloor\log_2 n\rfloor$ & A      & BK      & $O(\log_2 n)$ \\
Reverse chain       & $n{-}1$              & A or B & Parity  & $O(n)$ \\
Balanced binary     & $\lfloor\log_2 n\rfloor$ & B only & BinTree & $O(\log_2 n)$ \\
Balanced ternary    & $\lfloor\log_3 n\rfloor$ & B only & TerTree & $O(\log_3 n)$ \\
\bottomrule
\end{tabular}
\end{table}

\Cref{tab:encodings-as-trees} summarises the correspondence.  The
balanced binary and ternary trees are \emph{not} index-monotonic, so
Construction~A fails for them — only Construction~B is valid.  This
observation motivates the analysis of the structural phase boundary
in~\cref{sec:phase-boundary}.
