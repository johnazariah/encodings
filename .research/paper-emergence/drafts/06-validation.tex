% drafts/06-validation.tex — Empirical Validation
%
% Target length: ~2 pages (two-column PRA)
% Subsections: 6.1 Eigenspectrum, 6.2 Anticommutation,
%              6.3 Scaling, 6.4 Monotonicity census

All computations are performed using the open-source \textsc{FockMap}
library~\cite{fockmap2026}.

%%====================================================================
\subsection{Eigenspectrum equivalence}
\label{sec:val-eigen}

We encode the H$_2$ STO-3G Hamiltonian (4~spin-orbitals) with all five
encodings and verify that the resulting $16 \times 16$ qubit
Hamiltonians are unitarily equivalent.  All five yield ground-state
energy $E_0 = -1.1373\;\text{Ha}$ (exact full-CI) to machine
precision.

% TODO: Table of eigenvalues; reference MatrixVerification.fsx

%%====================================================================
\subsection{Anticommutation verification}
\label{sec:val-car}

For each encoding, we construct the full set of $2n$ Majorana operators
as $2^n \times 2^n$ matrices and verify all $\binom{2n}{2}$
anticommutators.

\begin{itemize}
  \item $n = 4, 8, 16$: all five path-based encodings satisfy the CAR.
  \item $n = 8$, balanced ternary tree with Construction~A:
    $\acomm{d_6}{d_7} \neq 0$, confirming the monotonicity violation
    (\cref{thm:monotonicity}).
\end{itemize}

% TODO: Table of diagnostic D(T) for each encoding/construction pair.

%%====================================================================
\subsection{Scaling benchmark}
\label{sec:val-scaling}

\Cref{tab:scaling} reports the maximum Pauli weight of any single
ladder operator for $n$ up to~24.

\begin{table}[t]
\centering
\caption{Maximum Pauli weight $\max_j \pauliw{\ad{j}}$ across encodings.}
\label{tab:scaling}
\begin{tabular}{@{}rrrrrr@{}}
\toprule
$n$ & JW & BK & Parity & BinTree & TerTree \\
\midrule
 4  &  4 &  4 &  4 &  3 &  3 \\
 8  &  8 &  4 &  8 &  5 &  3 \\
12  & 12 &  4 & 12 &  5 &  5 \\
16  & 16 &  5 & 16 &  5 &  5 \\
20  & 20 &  5 & 20 &  7 &  5 \\
24  & 24 &  5 & 24 &  7 &  5 \\
\bottomrule
\end{tabular}
\end{table}

A log-log fit confirms slopes consistent with $O(n)$ for JW/Parity,
$O(\log_2 n)$ for BK/BinTree, and $O(\log_3 n)$ for TerTree.

% TODO: Figure — log-log plot with fitted slopes.

%%====================================================================
\subsection{Monotonicity census}
\label{sec:val-census}

We enumerate all $n^{n-1}$ labelled rooted trees for small~$n$,
classify each as monotonic or non-monotonic, and compute the monotonic
fraction.

\begin{table}[t]
\centering
\caption{Monotonicity census.  $|M(n)|$ = number of monotonic trees
  out of $n^{n-1}$ total.}
\label{tab:monotonicity}
\begin{tabular}{@{}rrrc@{}}
\toprule
$n$ & $n^{n-1}$ & $|M(n)|$ & Fraction \\
\midrule
2 & 1      & 1     & 100\%  \\
3 & 9      & ?     & ?      \\
4 & 64     & ?     & ?      \\
5 & 625    & ?     & ?      \\
6 & 7776   & ?     & ?      \\
\bottomrule
\end{tabular}
\end{table}

For $n \ge 7$ we use random sampling ($10^5$ trees per~$n$) to
estimate the fraction.  Preliminary data support a rapidly vanishing
fraction consistent with \cref{conj:monotonic-fraction}.

% TODO: Fill table from MonotonicityCensus.fsx output.
% TODO: Figure — fraction vs. n with conjectured scaling fit.
