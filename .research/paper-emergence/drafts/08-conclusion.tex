% drafts/08-conclusion.tex — Conclusion
%
% Target length: ~0.5 page (two-column PRA)

We have shown that the fermion-to-qubit encoding is far more than a
compilation step: it is a minimal, exactly solvable model of quantum
emergence.

A single combinatorial object — a labelled rooted tree — completely
determines the encoding and, with it, every emergent property of the
qubit representation: locality structure (Theorem~\ref{thm:weight-bound}),
symmetry fractionalization (Theorem~\ref{thm:parity-weight}), gauge
constraints, and renormalization hierarchy.  None of these properties
exist in the fermionic description; they are created by the act of
re-representation.

The space of encodings itself possesses structure.  The sharp phase
boundary (Theorem~\ref{thm:monotonicity}) separating index-monotonic
from non-monotonic trees divides the encoding space into two
universality classes — one algebraic, one geometric — with the
algebraic phase becoming exponentially rare as $n$ grows.

All claims are computationally verifiable using the open-source
\textsc{FockMap} library, making this, to our knowledge, the simplest
and most calculable model of quantum emergence available.  The framework
invites further study: Hamiltonian-adapted trees, noise-aware
optimisation, composition with error correction, and the
characterisation of topological invariants in the encoding space are
all natural next steps.

The tree \emph{is} the encoding.  The encoding \emph{is} emergence.
