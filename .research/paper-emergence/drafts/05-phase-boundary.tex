% drafts/05-phase-boundary.tex — The Structural Phase Boundary
%
% Target length: ~3 pages (two-column PRA)
% Subsections: 5.1 Monotonicity, 5.2 Theorem 3, 5.3 Failure mode,
%              5.4 Phase diagram, 5.5 Universality classes

%%====================================================================
\subsection{The monotonicity property}
\label{sec:pb-mono}

\begin{definition}[Index-monotonic tree]
\label{def:monotonic}
A labelled rooted tree $T$ on $[n]$ is \emph{index-monotonic} if for
every vertex $v$ and every ancestor $u$ of $v$,
$\mathrm{label}(u) > \mathrm{label}(v)$.
\end{definition}

Equivalently, on every root-to-leaf path the labels strictly decrease.
The Fenwick tree is always index-monotonic; a balanced binary search
tree on $n = 8$ vertices is not (the root, labelled~4, has a child
labelled~6).

%%====================================================================
\subsection{The monotonicity constraint}
\label{sec:pb-theorem}

\begin{theorem}[Monotonicity constraint]
\label{thm:monotonicity}
Construction~A (the index-set method, \cref{sec:te-indexset}) produces
Majorana operators satisfying the CAR if and only if the tree $T$ is
index-monotonic.
\end{theorem}

\begin{proof}[Proof of necessity]
Suppose $T$ is non-monotonic: there exists a node~$j$ with an
ancestor~$a$ such that $a < j$.  In the computation of $\Rset{j}$,
node~$a$ contributes children with index $< j$ that lie on the
root-to-$j$ path.  This contaminates the $Z$~assignments in
$d_j$~\eqref{eq:dj-indexset}: the set
$(\Pset{j} \oplus \Occ{j}) \setminus \{j\}$ acquires nodes that also
appear in $\Uset{k}$ for some other mode~$k$, causing
$\acomm{d_j}{d_k} \neq 0$.

\textit{Explicit counterexample.}  Consider the balanced ternary tree
on $n = 8$ nodes.  Node~7 has parent~6, violating monotonicity
($6 < 7$).  A direct computation (detailed in
\cref{app:monotonicity-counterexample}) shows
$\acomm{d_6}{d_7} \neq 0$.
\end{proof}

\begin{proof}[Proof of sufficiency (sketch)]
If every ancestor of~$j$ has label $> j$, then $\Rset{j}$ contains
only nodes in sibling subtrees disjoint from the root-to-$j$ path.
The $Z$~assignments in $c_j$ and $d_j$ act on non-overlapping qubit
subsets for distinct modes, and anticommutation follows from the
orthogonality of these subsets.  A complete proof by induction on tree
depth is given in \cref{app:weight-proof}.
\end{proof}

%%====================================================================
\subsection{The failure mode}
\label{sec:pb-failure}

When monotonicity fails, the encoding breaks in a specific and
detectable way:
\begin{enumerate}
  \item The remainder set $\Rset{j}$ ``leaks'' onto the root-to-$j$
        path.
  \item The $Z$~assignments in $d_j$ interfere with those in $c_k$ for
        other modes.
  \item The anticommutation relation $\acomm{c_i}{d_j} = 0$
        ($i \neq j$) is violated.
  \item The encoded Hamiltonian has a \emph{wrong} eigenspectrum.
\end{enumerate}

This provides a computational diagnostic.  Given any tree and
Construction~A, one computes
\begin{equation}
  D(T) = \max_{i \neq j} \|\acomm{\an{i}}{\ad{j}}\|.
\end{equation}
If $D(T) = 0$ the CAR is satisfied; if $D(T) > 0$ the tree is
non-monotonic and only Construction~B may be used.

%%====================================================================
\subsection{The phase diagram}
\label{sec:pb-phase}

The space of labelled rooted trees on $[n]$ has cardinality $n^{n-1}$.
Let $M(n)$ denote the number of index-monotonic trees.

\begin{conjecture}
\label{conj:monotonic-fraction}
$|M(n)| / n^{n-1} \to 0$ as $n \to \infty$.  Specifically,
$|M(n)| / n^{n-1} = O(c^n / n!)$ for some constant $c$.
\end{conjecture}

As the system size grows, the fraction of encodings accessible to the
algebraic construction vanishes.  The ``generic'' encoding requires the
geometric (path-based) method.

% TODO: Figure — monotonic fraction vs. n from exhaustive enumeration
% (n = 2..6) and random sampling (n = 7..20).

%%====================================================================
\subsection{Two universality classes}
\label{sec:pb-classes}

The monotonicity boundary partitions encodings into two universality
classes:
\begin{description}
  \item[Algebraic (monotonic):] Compact index-set formulas,
    $O(\log n)$ per-operator computation via bit-twiddling.
    Fenwick tree (BK) is the canonical member.
  \item[Geometric (non-monotonic):] Explicit tree traversal required;
    more structural freedom, fewer constraints.
    Balanced ternary tree (optimal weight) lives here.
\end{description}

The path-based construction is \emph{universal} — it works in both
classes — but the index-set construction provides computational
shortcuts when monotonicity holds.  The geometric class may be viewed as
the ``disordered phase'' of the encoding space: fewer structural
constraints, richer phenomenology, but no compact algebraic shortcut.
